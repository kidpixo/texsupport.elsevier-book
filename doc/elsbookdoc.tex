\documentclass[a4paper,12pt]{article}
\usepackage{multicol}
\usepackage[T1]{fontenc}
\usepackage{newtxtext}
\usepackage{newtxmath}
\usepackage{fancyvrb}
\usepackage{enumitem}
\setlist%
{%
 topsep=0pt,%
 labelsep=6pt,
 noitemsep,%
 leftmargin=*
}
\setlist[description]{font=\normalfont\sffamily}

\title{elsevierbook.cls class}
\author{deimi}
\date{}

\def\file#1{\texttt{#1}}
\begin{document}
\maketitle

\section{Introduction}

\file{elsevierbook.cls} is designed for preparation \LaTeX{} books to be published by Elsevier. 
The document class is built on \file{book.cls} and require following packages:

\begin{itemize}
\item \file{etoolbox}
\item \file{calc}
\item \file{afterpackage}
\item \file{numname}
\item \file{xcolor}
\item \file{colortbl} for table with colored rows, columns;
\item \file{fontenc}
\item \file{textcomp} 
\item \file{amsmath} for math typesetting;
\item \file{amsmath,amssymb,amsfonts}
\item \file{amsthm} for theorem and alike;
\item \file{caption} for table, figure caption control;
\item \file{titlesec} for sectioning control;
\item \file{enumitem} for lists control;
\item \file{mdframed} for boxed text elements;
\item \file{footmisc} for footnote control;
\item \file{natbib} for bibliography; 
\item \file{minitoc} for producing chaper's minitoc;
\item \file{multicol}
\end{itemize}

\section{Installation}


The package is available at TeX Support page on github\\ (\file{https://github.com/vtex-soft/texsupport.elsevier-book})
It can also be found on CTAN (Comprehensive TeX Archive Network)
\file(http://www.ctan.org/tex-archive/macros/latex/contrib/elsevierbook/)
Please download the \file{elsevierbook.dtx} which is a class source file and
\file{elsevierbook.ins} which is a driver file for extraction the class file.
To produce \file{elsevier.cls} file, run LaTeX on the \file{elsarticle.ins}. 
The class file may be moved to a place, usually, \file(\$TEXMF/tex/latex/elsevierbook/), or a folder
that TeX can read  from. The TeX file database needs to be regenerated after moving the class file.
Usually, we use commands like \file{mktexlsr} or \file{texhash} depending upon the distribution and operating system. 

However, the installation is optional and you can skip this phase.
The bundle is self-contained and after unzipping your have everything you need for book preparation. (see \ref{bookstructrure})

\section{Book structure}\label{bookstructure}

Organising book can be challenging. 
Here we discribe two possible ways of book organisation.
The first one is pretty simple with flat folder structure that can suit for 
short books, the other is more complex that can serve for more complex books.

\subsection{Option 1 - simple stucture}
\begin{multicols}{2}
\begin{verbatim}
book.tex
chapter01.tex
chapter02.tex
dediction.tex
preface.tex
bibliography.tex
[...] 
img/
  ch01-figure01.eps
  ch02-figure02.eps
sty/
  elsevierbook.cls
\end{verbatim}

\columnbreak
\begin{verbatim}
book.tex: 
\documentclass{elsevierbook.cls}
\begin{document}
\Frontmatter
  %% Title Page %%
\begin{titlepage}%

  \title{Book title}
  \subtitle{Book Subtitle}
  \titlepagerule

  \edition{2nd edition}

  \author{Author Firstname  Surname}
  \address{Author Address}

  \author{Author Firstname  Surname}
  \address{Author Address}

\end{titlepage}

    %% Dedication page
\begin{dedication}
  %% Dedication
  %% can be few\\
  %% lines long
\end{dedication}

  \tableofcontents
  \cleardoublepage
\begin{frontmatter}
\chapter*{Preface}% 
\end{frontmatter}

This is an example input file.  Comparing it with
the output it generates can show you how to
produce a simple document of your own.

This is an example input file.  Comparing it with
the output it generates can show you how to
produce a simple document of your own.

This is an example input file.  Comparing it with
the output it generates can show you how to
produce a simple document of your own.

\source%
{%
  \author{Author Name}\\%
  % \address{Location}\\%
  % \date{Month Year}%
}

  \include{acknowledgement}
  [...]
\Mainmatter
  \include{chapter01}
  \include{chapter02}
  [...]
\Backmatter
  \begin{frontmatter}
\chapter{Appendix title}%\label{chap:appa}
\end{frontmatter}


%\section{}\label{}
  [...]
  \include{bibliograhy}
  \printindex
\end{document}
\end{verbatim}

\end{multicols}

\subsection{Option 2 - more complex structure}
For more complex projects this can be more suitable

\begin{multicols}{2}
\begin{verbatim}
book.tex
dediction.tex
preface.tex
bibliography.tex
[...]
chapter01/
  chapter01.tex
  img/
    ch01-figure01.eps
    ch02-figure02.eps
    [...]
chapter02/
  chapter02.tex
  img/
    ch02-figure01.eps
    ch02-figure02.eps
    [...]
[...]
appendix01/
  appendix01.tex
  img/
    appendix01-figure01.eps
    appendix01-figure02.eps
bibliography.tex
sty/
  elsevierbook.cls
\end{verbatim}

\columnbreak
\begin{verbatim}
book.tex: 

\documentclass{elsevierbook.cls}
\begin{document}
\Frontmatter
  %% Title Page %%
\begin{titlepage}%

  \title{Book title}
  \subtitle{Book Subtitle}
  \titlepagerule

  \edition{2nd edition}

  \author{Author Firstname  Surname}
  \address{Author Address}

  \author{Author Firstname  Surname}
  \address{Author Address}

\end{titlepage}

    %% Dedication page
\begin{dedication}
  %% Dedication
  %% can be few\\
  %% lines long
\end{dedication}

  \tableofcontents
  \cleardoublepage
\begin{frontmatter}
\chapter*{Preface}% 
\end{frontmatter}

This is an example input file.  Comparing it with
the output it generates can show you how to
produce a simple document of your own.

This is an example input file.  Comparing it with
the output it generates can show you how to
produce a simple document of your own.

This is an example input file.  Comparing it with
the output it generates can show you how to
produce a simple document of your own.

\source%
{%
  \author{Author Name}\\%
  % \address{Location}\\%
  % \date{Month Year}%
}

  \include{acknowledgement}
  [...]
\Mainmatter
  \include{chapter01}
  \include{chapter02}
  [...]
\Backmatter
  \begin{frontmatter}
\chapter{Appendix title}%\label{chap:appa}
\end{frontmatter}


%\section{}\label{}
  [...]
  \include{bibliograhy}
  \printindex
\end{document}
\end{verbatim}
\end{multicols}


\section{Usage}

The class should be loaded with the following command:

\begin{verbatim}
\documentclass[<options>]{elsevierboook}
\end{verbatim}

\noindent where the \verb!options! can be following:

\begin{description}[labelwidth=6pc,labelindent=0pc,labelsep=0pt,leftmargin=6pc]
\item[a02] sets \verb!a02! book model settings
\item[a08a] sets \verb!a08a! book model settings
\item[p05] sets \verb!p05! book model settings
\item[authoryear] for \file{natbib} package.
\end{description}



\section{Frontmatter}

 - chapter
 - author
 - address
 - footnotes
 - dispquotes
 - chapter points
 - minitoc
 - abstract
 - keywords

\section{Tables, Figures}

Figures may be included using the command \verb!\includegraphics!. Please check \verb!graphicx! package
for available options. Please use EPS format for figures  working with LaTeX, and PDF, PNG, MPS formats for pdfLaTeX. 
Do not use file extensions and path in order to load file. If you need LaTeX to find your graphics files in 
other folder than in setup, set the path into \verb!input@path!.

\begin{verbatim}
\begin{figure}
\includegraphics{file-name}% no path, no extension
\caption%
  {Figure caption %
     \source{Cortesy of [...]}% 
  }
\end{figure}
\end{verbatim}

Table environment may be enhanced depending on model chosen. 

\begin{verbatim}
\begin{table}
\begin{tableframe}
\caption{Table caption text [...]  }%
\begin{tabularx}{\textwidth}{X||X}
\Hline
\tch{Item A} & \tch{Item B}\tabnoteref{tn1}\\\hline
\tchi{Item A2} & \tchi{Item B2}\tabnoteref{tn1}\\\hline
\rowcolor{thd}
\multicolumn{2}{l}{\textbf{Item}} \\
Item A & Item B\\\hline
Item C & Item D\tabnoteref{tn2}\\\Hline
\end{tabularx}
\begin{tabnotes}
  \tabnotetext[*]{tn1}{Table footnote}
  \tabnotetext[a]{tn2}{Table footnote}
  \legend{EL=empirical likelihook.}
  \source{foo}
\end{tabnotes}
\end{tableframe}
\end{table}
\end{verbatim}

\begin{description}
\item[tableframe] - environment 
\item[source] - source 
\item[Hline] - heavy line
\item[tch] - table column head
\item[rowcolor{thd}] for colored rows
\item[tabnoteref] table footnote reference
\item[tabnotetext] table footnote text. Must be on \verb!tabnotes! environment
\item[legend] table legend. Must be inside \verb!tabnotes! environment.
\end{description}


\section{Boxed text}

Boxed text environments uses mdframed package as its basic. There are two types defined: Box Type A (BtypeA) and 
Box Type B (BtypeB)

Unnumbered text boxes
\begin{Verbatim}
\begin{textbox}[style=BtypeA,frametitle={Box type A}]
 Some text [...]
\end{textbox}
[...]
\begin{textbox}[style=BtypeB,frametitle={Box type B}]
 Some text [...]
\end{textbox}
\end{Verbatim}

\noindent Numbered text boxes are defined pretty much the same as theorem like environments.
\begin{Verbatim}
\mdtheorem[style=BtypeA]{example}{Example}[chapter]
\mdtheorem[style=BtypeB]{boxb}{Box}
\begin{example}[Numbered Box type A ]
  Some text [..]
\end{example}
[...]
\begin{boxb}[NUmbered Box type B]
  Some text [..]
\end{boxb}
\end{Verbatim}

You can use other options from mdframed package to fine tune the \verb!textbox! environment.

\section{Theorems and friends}

The class loads \verb!amsthm! package to make it easier to define theorem environments and the alike.
\begin{verbatim}
\newtheorem{theorem}{Theorem}
\theoremstyle{definition}
\newtheorem{definition}{Definition}
\theoremstyle{remark}
\newtheorem{remark}{Remark}
\end{verbatim}

Please refer to \verb!amsthm! package documentation for details.


\section{Lists}

We use \verb!enumitem! package for enumerate, itemize and other lists environments. It is possible 
to supply optional arguments to fine control the appearance of list.

\begin{Verbatim}
\begin{enumerate}[<options>]
\item [..]
\end{enumerate}
\end{Verbatim}

Please check \verb!enumitem! documentation for details.


\section{Display mathematics}

The package \file{amsmath} is loaded by the class file. You can use all environments form amsmath package. 
Please do not use faulty 'eqnarray' environment, but eqnalign or .. instead.


\section{Program Lists}

not available yet.

\end{document}

\section{Cross-references}


\section{Bibliography}

\section{Acknowledgement}

\section{Appendices}

\section{Index}

\section{Glossary List}

\section {Book Frontmatter}

\subsection{titlepage}
\subsection{Preface}
\subsection{About Author}
\subsection{Contributors Lists}
\subsection{Toc, lof, lot, etc}
\subsection{Dedication}



\end{document}